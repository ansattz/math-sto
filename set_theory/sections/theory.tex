\section{Teoria Axiomática [Conjuntos]}
   A teoria dos tipos resolve o paradoxo de Russell. No entanto, faz isso 
   com algum custo... a teoria de Russell nos dá múltiplas noções de conjuntos.
   Então, você teria problemas ao perguntar algo do tipo... "O que é um 
   conjunto?" A resposta então seria que existem tipos de conjuntos... 
   tipo zero, tipo um, ...\newline

   Eis que surge a abordagem axiomática para responder aos paradoxos da teoria 
   dos conjuntos ingênua. A abordagem foi iniciada por Ernst Zermelo em 1908 e 
   desenvolvida por Abraham Fraenkel em 1922. O que resulta é o sistema de 
   axiomas mais comum: a teoria dos conjuntos de Zermelo-Fraenkel. (Teoria 
   ZF).
   \subsection{Teoria ZF}
      Condições estabelecidas da teoria:
      \begin{enumerate}
         \item Todos os objetos da teoria são conjuntos. Dizemos que \emph{todos os 
            elementos de conjuntos serão conjuntos}, e assim sucessivamente, mas paramos 
            ao chegarmos no conjunto vazio.
        \item Os objetos da teoria não comportarão átomos e nem conjuntos de todos 
           os conjuntos (conjuntos universais), isto é, não comportarão 
           \emph{urelementos}.
      \end{enumerate}
      A estrutura do método axiomático se estabelece da seguinte forma:
      \begin{enumerate}
         \item Linguagem
         \item Axiomas
         \item Teorema - consequência dos axiomas
         \item Regras de inferência da \emph{lógica clássica}
      \end{enumerate}

      \subsubsection{Linguagem}
         A linguagem é dividida em 3 segmentos
         \begin{center}
            \textbf{Variáveis}\quad \textbf{Símbolos}\quad \textbf{Fórmulas} 
         \end{center}
         \begin{enumerate}[a)]
            \item Variáveis: Utilizado tanto letras minúsculas quanto maiúsculas para denotar conjuntos.
            \item Símbolos:  
               \subitem $\forall x$: para todo conjunto x...
               \subitem $\exists x$: existe um conjunto x...
               \subitem $\neg$: negação de...
               \subitem $\&$: e
               \subitem $\lor$: ou (acomete uma disjunção)
               \subitem $\rightarrow$: se... então (condicional)
               \subitem $\leftrightarrow$: se e somente se (bicondicional)
            \item Fórmulas: Sequências de símbolos que satisfazem os seguintes itens
               \subitem Se $x$ e $y$ são variáveis, então $x \in y$\ e\ $x = y$ são fórmulas. \emph{Fórmulas atômicas};
               \subitem Se $A$ e $B$ são fórmulas, então $\neg(A), (A) \rightarrow (B), (A) \& (B), (A) \lor (B)$\ e\ $(A) \leftrightarrow (B)$ também são;
               \subitem Se $A$ é uma fórmula e $x$ uma variável, então $\forall x(A)$\ e\ $\exists x(A)$ também são fórmulas;
               \subitem Todas as fórmulas tem uma das formas descritas acima.
         \end{enumerate}
         \begin{exmp}
            A análise da composição de fórmulas de
            $$\exists B(\forall x(\neg(x \in B))$$
            se da por
            \begin{enumerate}[I.]
               \item Existência da fórmula atômica\ $[(x \in B)]$.
               \item Uma outra fórmula está sendo a negação da fórmula atômica\ $[(\neg(x \in B)]$
               \item Fórmula quantificando a variável $x$ em relação a negação da fórmula atômica\ $[\forall x()]$
            \end{enumerate}
            Portanto, temos 3 fórmulas que são $$\neg(x \in B)),\quad \forall x(\neg(x \in B)) \quad e\quad \exists B (\forall x(\neg(x \in B)).$$
            \emph{"Existe um objeto $B$ tal que para todo objeto $x$, não é verdade que $x$ pertence a $B$}."
         \end{exmp}
         \begin{mymdframed}{Observação (Fórmulas lógicas)}
            \begin{itemize}
               \item\textbf{\underline{Ocorrência de variável:}} cada aparecimento de uma variável.
                  Sendo $$x \in y\ \&\ \exists x (y = x)$$
                  temos duas ocorrências da variável $y$ e duas\footnote{Alguns autores colocam o $x$, em $\exists x$, como uma ocorrência.} ocorrências da variável $x$.
               \item \textbf{\underline{Escopo de uma variável:}} uma variável $y$ está no escopo de uma variável $x$ se $y$ aparece em uma fórmula do tipo $\forall x (\ \ )$\ ou \ $\exists x (\ \ )$.
                  Sendo $$x \in y\ \& \ \exists x (y = x)$$
                  $y$ está no escopo de $x$ e $x$ está no escopo de si mesmo.
               \item \textbf{\underline{Variável livre:}} uma variável é livre se não está no escopo dela mesma.
                  Sendo $$x \in y\ \& \ \exists x (y = x)$$
                  $x$ e $y$ que estão fora da sub-fórmula são variáveis livres. $y$ que está dentro da sub-fórmula é uma variável livre, mas, dentro da sub-fórmula, $x$ não é uma variável livre por está dentro do próprio escopo.
            \end{itemize}
         \end{mymdframed}

   \subsection{Axiomas}
      \begin{itemize}
         \item Axiomas que garantem a existência de certos conjuntos específicos
            \subitem Axioma do vazio
            \subitem Axioma do infinito    
         \item Axiomas que nos permitem construir novos conjuntos
            \subitem Axioma do par
            \subitem Axioma da união
            \subitem Axioma das partes
            \subitem Axioma da escolha
            \subitem Axioma da separação
            \subitem Axioma da substituição
         \item Axiomas que dizem respeito ao "comportamento" dos conjuntos
            \subitem Axioma da regularidade 
            \subitem Axioma da extensão
      \end{itemize}

      \subsubsection{Axioma da extensão (ou da unicidade)}
         \begin{stat}
            Se dois conjuntos $x$ e $y$ possuem exatamente os mesmos elementos, então eles são iguais.
         \end{stat}
         $$\forall x\ \forall y (\forall z (z \in x\ \leftrightarrow \ z \in y)\ \rightarrow x = y)$$
         \begin{exmp}
            Se $x = \{1, 2, 3\}$, $y = \{3,1,2\}$ e $z = \{1,1,2,3,3,3\}$, então $x = y = z$.
         \end{exmp}
         
      \subsubsection{Axioma do vazio}
         \begin{stat}
            Existe um conjunto sem elemento algum.\footnote{Posteriormente, surgirá a necessidade de verificar a dispensabilidade desse axioma.}
         \end{stat}    
         $$\exists x \forall y\ \neg(y \in x)$$
         Utiliza-se $\varnothing$ para denotar este conjunto do Axioma do vazio. Para isto, devemos assegurar que
         \begin{enumerate}[a.]
            \item Existe um conjunto com tal propriedade;
            \item Esse conjunto é único (não pode-se atribuir o mesmo símbolo para duas coisas distintas).
         \end{enumerate}
         \begin{definition}[Relação de inclusão]
            Dizemos que $x$ está contido em $y$ e escrevemos $x \subseteq y$ se todo elemento de $x$ é um elemento de $y$. 
            $$ x \subseteq y\ \ \leftrightarrow\ \ \forall z (z \in x\ \ \rightarrow\ \ z \in y)$$
         \end{definition}
         Com essa definição podemos reescrever o \textbf{Axioma da extensão}: $$(A \subseteq B)\ \&\ (B \subseteq A)\ \ \rightarrow\ \ A = B$$
         Ainda, temos $x \subset y$ para dizer que $x \subseteq y\ \ \&\ \ x \neq y$.
         \begin{theorem}\label{T1}
            O conjunto vazio está contido em qualquer conjunto.
            $$\forall x (\varnothing \subseteq x)$$
            \begin{proof}
               Por contradição vamos supor que exista um conjunto x tal que o conjunto vazio não esteja contido em x e que existe y tal que y pertence ao conjunto vazio mas não a x. Com efeito:
               $$\exists x(\varnothing \nsubseteq x)\quad \&\quad \exists y(y \in \varnothing\ \&\ y \notin x).$$
               Temos uma contradição à própria definição de conjunto vazio.
            \end{proof}
         \end{theorem}
         \begin{theorem}\label{T2}
            Existe um único conjunto vazio.
            \begin{proof}
               A existência é garantida pelo Axioma do vazio. Com efeito, supondo que existam conjuntos vazios $x$ e $y$, com $x \neq y$. Pelo axioma da extensão deve então existir um elemento $z$ tal que
               \begin{alignat*}{3}
                  & z \in x  \quad & mas \quad & z \notin y\quad ou\\
                  & z \in y  \quad & mas \quad & z \notin x.
               \end{alignat*}
               Em qualquer caso temos uma contradição com o fato de que $x$ e $y$ são vazios.
            \end{proof}
         \end{theorem}
         \begin{mymdframed}{Observação (Relação de inclusão vs. Relação de pertinência)}
            \begin{itemize}
               \item Todos os objetos da nossa teoria são conjuntos. Portanto, nas expressões $x \in y$\ \ e\ \ $x \subseteq y$, tanto $x$ quanto $y$ são conjuntos;
               \item Podem existir conjuntos cujos elementos são conjuntos.
            \end{itemize}
            Supondo que exista o conjunto $\{\varnothing\}$, ou seja, um conjunto cujo único elemento é $\varnothing$.
            \begin{itemize}
                  \item $\varnothing \in \{\varnothing\}$, isto é, $\varnothing$ é elemento do conjunto $\{\varnothing\}$
                  \item $\varnothing \subseteq \{\varnothing\}$, de acordo com o Teorema\ref{T1}.
            \end{itemize}
         \end{mymdframed}
      
      \subsubsection{Axioma do Par}
         \begin{stat}
            Sejam $x$ e $y$ conjuntos quaisquer. Então existe um conjunto $x \in z$\ \ e\ \ $y \in z$.
         \end{stat}
         $$\forall x, y\ \exists z (x \in z\ \&\ y \in z)$$
         \begin{center}
            Aqui, dizemos que $z$ é um par não ordenado, pois, pelo axioma da extensão, \\
            os conjuntos $\{x, y\}$\ e\ $\{y, x\}$\ são iguais.
         \end{center}
         \begin{exmp}
            No caso em que $x=y$ temos $\{x, x\} = \{x\}$, pelo axioma da extensão.
         \end{exmp}

      \subsubsection{Axioma da União}
         \begin{stat}
            Para todo conjunto $A$, existe um conjunto $B$ cujos elementos são exatamente os elementos dos elementos de $A$.
         \end{stat}
         B será denotado por $\bigcup\limits_{X \in A}^{} A$
         \begin{itemize}
            \item todo $X$, que é elemento de $A$, é subconjunto de $B$
            \item todo $Y$, que é elemento de $B$, é elemento de algum elemento de $A$
         \end{itemize}
         Neste axioma temos também que
         \begin{itemize}
            \item Caso $A$ seja um vazio, $B$ será o conjunto vazio
            \item Caso $A$ seja finito, $B$ será uma união finita
            \item Caso $A$ seja infinito, $B$ será uma união infinita.
         \end{itemize}
         \begin{definition}
            A \underline{união generalizada} de um conjunto $A$ é o conjunto formado pelos elementos dos elementos de $A$.
            $$\forall A\,\exists B\,\forall x\,(x\in B\ \leftrightarrow\ \exists c\,(c\in A\ \&\ x\in c))$$
         \end{definition}
         \begin{exmp}
            Seja um conjunto $A = \{\{1,\ 2\},\ \{3,\ 4\},\ \{5,\ 6\}\}$.\\
            Então, a união generalizada se realiza como $\bigcup A = \{1,\ 2,\ 3,\ 4,\ 5,\ 6,\}$.\\
            O conjunto $A$ possui 3 elementos que são conjuntos que possuem elementos que estão em $A$.
         \end{exmp}

      \subsubsection{Axioma das Partes} 
         \begin{stat}
            Para todo conjunto $x$, existe o conjunto $y$ dos subconjuntos de $x$.
            $$\forall x \exists y \forall z [z \in y \leftrightarrow z \subseteq x]$$
         \end{stat}
         O conjunto definido pelo axioma também é único. Sendo assim, podemos definir o seguinte:
         \begin{definition}
            Definimos o \emph{conjunto das partes} (\emph{power set}), como o conjunto de todos os subconjuntos de $x$, e o denotamos por $\mathcal{P}(x)$.
         \end{definition}
         \begin{exmp}
            Seja um conjunto $A = \{1,2\}.$\\
            Pelo axioma das partes, $$(z \in  \mathcal{P}(A)\ \leftrightarrow\ z \subseteq A)$$
            \begin{alignat*}{3}
               & \varnothing \subseteq A\\
               & \{1\} \subseteq A\\
               & \{2\} \subseteq A\\
               & \{1,2\} \subseteq A.
            \end{alignat*}
            Logo, $\mathcal{P}(A) = \{\varnothing,\ \{1\},\ \{2\},\ \{1,\ 2\}\}$
         \end{exmp}

      \subsubsection{Axioma da Separação}
         Queremos ser capazes de definir um conjunto através de uma fórmula lógica. Para evitar paradoxos, como o paradoxo de Russel, temos que especificar o conjunto sobre qual nossa fórmula será elaborada.
         \begin{stat}
            Para cada fórmula $P(x)$, para todo conjunto $y$, existe o conjunto daqueles elementos que pertencem a $y$ e satisfazem a fórmula $P$.
         \end{stat}
         \begin{exmp}
            Seja um conjunto $y = \{1,2,3,4,5,6,7,8,9\}$.

            $P(x): x$ é par.

            $Z = \{x: x \in y\ \&\ P(x)\} = \{2,4,6,8\}$
         \end{exmp}
         \begin{definition}
            Para cada fórmula $P$ tal que $z$ não ocorre livre, a seguinte fórmula é um axioma:
            $$\forall y \exists z \forall x [x \in z\ \leftrightarrow\ x \in y\ \&\ P(x)]$$    
         \end{definition}
         \emph{"para todo $y$ existe $z$ tal que para todo $x$, $x$ pertence a $z$ sss $x$ pertence a $y$ e vale a fórmula $P(x)$."}\\
         O conjunto $z$ será denotado
         $$z = \{x: x \in y\ \&\ P(x)\}$$
         ou
         $$z = \{x \in y: P(x)\}.$$
         Por que $z$ não pode ocorrer \underline{livre} em $P$ na fórmula
         $$\forall y \exists z \forall x [x \in z\ \leftrightarrow\ x \in y\ \&\ P(x)] ?$$
         Se permitirmos que a variável que \underline{define} o conjunto ocorra livre em $P$, poderíamos tomar \\ $P(x) = \highl{x \notin z}$:
         $$\forall y \exists z \forall x [x \in z\ \leftrightarrow\ x \in y\ \&\ \highl{x \notin z}\color{black}].$$
         Tomando $y = \{\varnothing\}$\ e\ $x = \varnothing$, temos que $x \in y$ é verdadeiro. Assim,
         $$(x \in z\ \leftrightarrow\ \highl{x \notin z} \color{black})$$
         é uma \underline{contradição}. 

      \subsubsection*{Interseção}
         Podemos utilizar o axioma da separação para definir a interseção de dois conjuntos.

         Sejam $A$ e $B$ dois conjuntos. Então, pelo axioma da separação existe 
         $$z = \{x: x \in A\ \&\ P(x)\}.$$
         Tomando $\color{brown}{P(x) = x \in B}$, temos
         $$\color{violet}{z = \{x: x \in A\ \&\ x \in B \} = A \cap B}.$$

      \subsubsection*{Diferença e conjuntos vazios}
         Da mesma forma, podemos definir a diferença entre dois conjuntos.

         Sejam $A$ e $B$ dois conjuntos. Então, pelo axioma da separação existe 
         $$z = \{x: x \in A\ \&\ P(x)\}.$$
         Tomando $\color{brown}{P(x) = x \notin B}$, temos
         $$\color{violet}{z = \{x: x \in A\ \&\ x \notin B \} = A - B}.$$
         Se tomarmos $\color{brown}{P(x) = x \neq x}$, obtemos o conjunto vazio.

      \subsubsection*{Interseção arbitrária de conjuntos}
         \begin{theorem}
            Dado um conjunto não vazio $x$, existe o conjunto $y = \bigcap x$, formado por todos os elementos que pertencem simultaneamente a todos os elementos de $x$.
            $$\bigcap x = \{v: v\ \textrm{pertence a todos os elementos de}\ x\}.$$
         \end{theorem}
         \begin{exmp}
            Seja um conjunto $x = \{\{1,2,3\},\ \{2,3,4\},\ \{2,3,4,5\}\}.$\\
            Existe o conjunto interseção: $\bigcap x = \{2,3\}.$
            Formalmente, temos $$y = \{v: v \in z\ \&\ \forall w (w \in x\ \rightarrow\ v \in w)\}.$$
         \end{exmp}

      \subsubsection{Axioma da regularidade}
         \begin{stat}
            Para todo conjunto $x$ não vazio existe $y \in x$ tal que $x \cap y = \varnothing$.
            $$ \forall x [x \neq \varnothing\ \rightarrow\ \exists y (y \in x\ \&\ x \cap y = \varnothing)].$$
         \end{stat}
         Ou seja, esse axioma garante que não exista uma sequência infinita decrescente na relação de pertinência. Não há como ficar preso numa sequência infinita decrescente, pois certa hora irá encontrar o conjunto vazio.
         \begin{theorem}
            Não existem conjuntos $x$ e $y$ tais que $x \in y$\ e\ $y \in x$.
         \end{theorem}
         \begin{proof}
            Sejam $x$ e $y$ conjuntos quaisquer.
            Vamos demonstrar que ou $x \notin y$\ ou\ $y \notin x$.\\
            Pelo \textbf{axioma do par}, existe $B = \{x,y\}$ não vazio.\\
            Pelo \textbf{axioma da regularidade}, $B \neq \varnothing\ \rightarrow\ \exists z (z \in B\ \&\ z \cap B = \varnothing).$\\
            Mas $z \in B$ implica que $z = x$\ ou \ $z = y$.\\            
            Se $z=x$, então $z \cap B\ =\ x \cap B\ =\ \varnothing$, logo, $y \notin x$,\\
            (pois caso contrário, o resultado da interseção conteria o $y$).\\
            Se $z=y$, então $z \cap B\ =\ y \cap B\ =\ \varnothing$, logo, $x \notin y$.
         \end{proof}
         \begin{mymdframed}{Nota}
            A formulação dada por von Neumann (1925) em lógica de primeira ordem é seguida por
            $$\forall A(\exists B(B\in A)\rightarrow \exists B(B\in A\land \neg \exists C(C\in A\land C\in B)))$$
            \emph{Todo conjunto diferente do conjunto vazio possui um elemento que é totalmente disjunto dele.}\\
            Sem esse axioma o comportamento possível de infundados conjuntos surgem. São denominados \textbf{hiper-conjuntos}. Um comportamento típico seria $y \in y$, então $y$ é um hiper-conjunto. Em particular, não é adequado ter a violação desse axioma, apenas para versões da teoria dos conjuntos que possam se beneficiar disso.
         \end{mymdframed}
