\section{Construção}
   Há duas maneiras de introduzir novos objetos em matemática:
   \begin{itemize}
      \item Axiomaticamente
      \item Construtivamente
   \end{itemize}
   O primeiro diz respeito a tomarmos o conceito conjunto como 
   primitivo e importamos axiomas para tratar deste conceito. 
   O segundo definimos o novo objeto em termos de objetos ja estabelecidos.
   Pela forma construtiva, lidaremos com numeros como conjuntos.

   \subsection{Construção de von Neumann}
      A construção de von Neumann para os numeros naturais 
      \emph{considera um numero natural como o conjunto dos numeros 
      menores a esse numero}.
      Essa forma construtiva concebe um numero natural como um conjunto dos numeros 
      menores a esse numero:
      \begin{alignat}{3}
         \nonumber 0 &:=& \varnothing\\
         \nonumber 1 &:=& \{0\} &:= \{\varnothing\}\\
         \nonumber 2 &:=& \{0,1\}& := \{\varnothing, \{\varnothing\}\}\\
         \nonumber 3 &:=& \{0,1,2\} &:= 
         \{\varnothing, \{\varnothing\}, \{\varnothing, \{\varnothing\}\}\}\\
         \nonumber &\vdots
      \end{alignat}
      Com isso temos: $0 \in 1 \in 2 \in 3 \in ...$, assim 
      como $ 0 \subseteq 1 \subseteq 2 \subseteq 3 \subseteq ...$. 
      Usando a ideia de pertinencia da teoria dos conjuntos podemos 
      estabelecer uma noção de ordem.

      \subsubsection{Sucessor de um conjunto}
         Vamos buscar uma definição geral para dizer que algo é um número natural.
         Da construção de von Neumann temos o seguinte:
         $$3 = \{0,1,2\} = \{0,1\} \cup \{2\} = 2 \cup \{2\}$$
         $$2 = \{0,1\} = \{0\} \cup \{1\} = 1 \cup \{1\}$$
         Isso significa que o número natural seguinte a $n$ pode ser obtido pela união 
         de $n$ com $\{n\}$.
         \begin{definition}
            Para qualquer conjunto $n$, o \emph{sucessor} de $n$, 
            denotado por $n++$, é o conjunto $$n++ = n \cup \{n\}$$
            i.e, $$\forall y(y \in n++ \leftrightarrow y \in n \lor y = n).$$
         \end{definition}
         \begin{exmp}
            $$2 \cup \{2\} = 2++ = 3.$$
         \end{exmp}
         \begin{definition}[Conjunto indutivo]
            Um conjunto $A$ é dito \emph{indutivo} se e somente se
            \begin{itemize}
               \item $\varnothing \in A$
               \item $A$ é fechado sob a operação sucessor, i.e, 
                  $\forall n (n \in A \rightarrow n++ \in A).$
            \end{itemize}
         \end{definition}
         Em outras palavras, o conjunto indutivo é um conjunto que possui 
         como elemento o conjunto vazio e para qualquer objeto que 
         pertence a esse conjunto o sucessor também pertence ao conjunto.\\
         Nesse sentido podemos escrever os números naturais em termos 
         de conjunto vazio e da operação de sucessor:
         \begin{alignat}{3}
            \nonumber 0 &=& \varnothing ++\\
            \nonumber 1 &=& 0++\\
            \nonumber 2 &=& 1++\\
            \nonumber 3 &=& 2++
         \end{alignat}
         Temos que cada número é distinto (satisfazendo o axioma da extensão). 
         O importante dessa exposição é que apesar de termos infinitos 
         números (conjuntos), todos os números naturais são conjuntos finitos. 
         Ainda não temos um axioma que nos permite dizer que existe algum conjunto
         infinito (ideia intuitiva sobre os naturais), assim, não podemos 
         estabelecer a existência de um conjunto indutivo.

   \subsection{Axioma do infinito e os números naturais}
      \begin{stat}[Axioma do Infinito]
         Existe um conjunto indutivo, i.e:
         $$\exists A\left(\varnothing \in A\ \&\ 
         \forall n(n \in A \rightarrow n ++ \in A)\right).$$
      \end{stat}
      \emph{Existe um conjunto $A$ tal que o conjunto vazio pertence ao 
         conjunto $A$ e para todo $n$ que pertence a $A$ o sucessor de $n$
      pertence a $A$.}\\
      Obviamente que um conjunto indutivo contém necessariamente todos os
      números naturais. Outra noção é que ele pode conter outros conjuntos. 
      E é através dessa noção que intuitivamente temos que o conjunto dos 
      números naturais é o menor conjunto indutivo.

      \subsubsection{Conjunto dos números naturais}
         A coleção de todos os números naturais é um conjunto.
         \begin{theorem}
            Existe um conjunto $\omega$ cujos elementos são 
            exatamente os números naturais.
            \begin{proof}
               Seja $A$ um conjunto indutivo assegurado pelo axioma do infinito.
               Pelo axioma da separação existe o conjunto
               $$\omega = \{n:n \in A\ \&\ n\ 
               \textrm{pertence a todo conjunto indutivo}\}$$
               Como $n$ pertence a todo conjunto indutivo então ele pertence a 
               $A$ e, portanto, $$\omega = \{n: n\ 
               \textrm{pertence a todo conjunto indutivo}\}$$
               Pela definição de número natural, segue então que
               $$\omega = \{n: n\ \textrm{é um número natural}\}$$
            \end{proof}
         \end{theorem}
         \begin{theorem}[Unicidade do conjunto omega]
            Seja $\omega = \{n: n\ \textrm{é um natural}\} = 
            \{n: n\ \textrm{pertence a todo conjunto indutivo}\}$
            \begin{enumerate}[i.]
               \item $\omega$ é indutivo.
               \item $\omega$ é subconjunto de todo conjunto indutivo,
                  i.e, se $A$ é indutivo temos que $\omega \subseteq A$.
               \item $\omega$ satisfazendo i. e ii. é único.
            \end{enumerate}
            \begin{proof}
                i. $\varnothing \in \omega$, pois $\varnothing$ pertence a 
                todo conjunto indutivo.\\
                Se $n \in \omega$, então $n$ pertence a todo conjunto indutivo.\\
                Portanto, $n++$ pertence a todo conjunto indutivo e $n++ \in \omega$.
            \end{proof}
            \begin{proof}
                ii. - iii. Seja $A$ um conjunto indutivo. Considere $n\in \omega$.\\
                Pela definição, $n$ pertence a todo conjunto indutivo, 
                em particular, $n \in A$.\\ 
                Mostramos que $n \in \omega \rightarrow n \in A$. 
                Logo, ii. é válido.\\
                Suponha que exista outro conjunto $B$ além de $\omega$
                que satisfaz i. e ii.\\
                Assim, devemos ter, por ii., $B\subseteq \omega\ \&\
                \omega \subseteq B$. Pelo axioma da extensão, $B = \omega$.
            \end{proof}
         \end{theorem}
         O teorema anterior nos diz que o conjunto $\omega$ é o menor conjunto indutivo.
         Qualquer subconjunto indutivo de $\omega$ coincide com $\omega$.
         \begin{theorem}
            Todo número natural, exceto $0$, é sucessor de algum número natural.
            \begin{proof}
               Pelo axioma da separação, definimos um subconjunto de $\omega$
               cujos elementos satisfazem a hipótese do teorema. O subconjunto possui
               a seguinte configuração
               $$A = \{n \in \omega: n = 0 \lor \exists s \in \omega (n = s++)\}.$$
               Queremos mostrar que $A = \omega$, i.e, que 
               todos os números satisfazem o teorema.\\
               Basta mostrar que $A$ é indutivo:
               \begin{enumerate}[i.]
                  \item $0 \in A$. Isso é imediato porque $0 \in \omega$.
                  \item $k \in A \rightarrow k++ \in A$.
               \end{enumerate}
               Para $k \in A$ significa que $k \in \omega$, $k = 0$ ou que 
               $\exists s \in \omega(k = s++)$.\\
               Se $k = 0$: caso base.\\
               Como provamos que $\omega$ é indutivo, segue que $s++$ 
               pertence a $\omega$.\\
               Por hipótese se $k$ pertence a $\omega$ o sucessor de 
               $k$ também pertence a $\omega$.\\
               Assim, $k++ = (s++)++$ e segue que $k++ \in A$. 
               Logo, $A$ é indutivo.\\
               Portanto, pelo princípio da indução, $A = \omega$.
            \end{proof}
         \end{theorem}
   
   \subsection{Axiomas de Peano}
      \begin{enumerate}[1.]
         \item Zero é um número.
         \item O sucessor de um número é um número.
         \item Números distintos nunca têm o mesmo sucessor.
         \item Zero não é sucessor de qualquer número.
         \item Se uma propriedade vale para zero, e, valendo para um
            número $n$ também vale para seu sucessor, então, essa
            propriedade vale para todos os números.
      \end{enumerate}
      \begin{theorem}
         O conjunto $\omega$ satisfaz os axiomas de Peano.
         A correspondência é o número como elemento de $\omega$, zero
         como o conjunto vazio de $\omega$ e sucessor como o conjunto 
         $\delta$ para $\omega$ definido como 
         $$\delta = \{(n, n++): n \in \omega\}.$$
         \emph{O conjunto delta é o conjunto dos pares ordenados 
         n e o sucessor de n tal que n está em $\omega$}.
         \begin{proof}
            1. e 2.\\
            Estes dois axiomas são satisfeitos porque $\omega$ é indutivo.
         \end{proof}
         \begin{proof}
            3.\\
            Vamos supor que $n \neq s$, mas $n++ = s++$.
            Como $n \in n++$ segue que $n \in s++$, o que é o mesmo que
            dizer que $n \in (s \cup \{s\}).$\\
            Ou seja, $n\in s \lor n = s$. Por hipótese $n \neq s$.
            Portanto $n \in s$. Analogamente para $s \in n$.\\
            Isso contradiz o Teorema 4 do Axioma da Regularidade:\\
            \emph{Não existem conjuntos $x$ e $y$ tais que 
            $x \in y$ e $y \in x$.}
         \end{proof}
         \begin{proof}
            4.\\
            Temos a informação de que $\forall n (n \in n++).$ Logo, não
            podemos ter $n++ = 0 = \varnothing$, já que o conjunto vazio
            não possui elementos.
         \end{proof}
      \end{theorem}
         \begin{proof}
            5.\\
            Princípio da indução. Para $\omega$ obtivemos a indução
            considerando um subconjunto $A$ e descobrindo que se $A$
            é indutivo, então $A$ é igual ao conjunto $\omega$.
         \end{proof}
      \subsubsection{Sistema de Peano}
         $$(N, S, i)$$
         $N$ o conjunto indutivo, $S$ a operação sucessora e $i$ o objeto que
         pertence ao conjunto $N$. Esse sistema é isomórfico ao sistema
         $$(\omega,\delta,0).$$ Existe uma função bijetora $$h: \omega \to N$$
         que preserva a operação de sucessor e o primeiro elemento.
