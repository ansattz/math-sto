\section{Introdução}
   \subsection{Conjuntos ingênuos}
      Um conjunto, de uma perspectiva ingênua, é apenas uma coleção 
      de coisas que você queira considerar; algo que você queira 
      colecionar. Geralmente as primeiras ideias de coleção são 
      conjuntos de números. Mas essa perspectiva é expansiva, 
      isto é, pode-se considerar o conjunto de todas as estrelas 
      do universo e o conjunto de todos os animais da terra, 
      mas a coleção de estrelas e animais formam um conjunto.\newline
      As primeiras pistas de que há um problema com a noção ingênua de 
      conjuntos vieram com o relato de Georg Cantor sobre 
      \emph{conjuntos infinitos}. Se podemos coletar qualquer coisa 
      para fazer um conjunto, então por que não coletar todos os 
      conjuntos? Então formamos o conjunto de todos os conjuntos.\newline
      Essa noção de conjunto ficou evidentemente conhecida como ingênua 
      quando abrigou um paradoxo que inibiu o andamento da teoria dos 
      conjuntos desde o início do século XX. Esse é o \emph{paradoxo de 
      Russell}. 

   \subsection{Paradoxo de Russell}
      O paradoxo de Russell surge de uma variação no problema do conjunto 
      de todos os conjuntos. Tudo o que é necessário é a ideia ingênua de um 
      conjunto e a noção do que é para algo ser um elemento (ou membro, 
      como disse Russell) de um conjunto.
      \begin{center}
         \emph{Existem conjuntos que são membros de si mesmos e existem 
         conjuntos que não são membros (elementos) de si mesmos.}
      \end{center}
      A importância do paradoxo foi profundamente destrutiva. Russell vinha 
      seguindo o trabalho de Gottlob Frege nos fundamentos lógicos da aritmética. 
      Russell, com seu famoso paradoxo, atinge o "coração" do sistema de Frege.
      Russell explicou a dificuldade a Frege em uma carta de 1902. Frege 
      respondeu e reconheceu a gravidade do problema. Chegou em um momento de 
      dificuldade. Frege tinha acabado de completar o segundo volume de uma 
      grande obra, Grundgesetze der Arithmetik [Leis Básicas da Aritmética]. 
      Ele conseguiu adicionar um apêndice à impressão.
      \begin{Center}
         \textit{"Dificilmente algo mais infeliz pode acontecer a um escritor 
         científico do que ter uma das fundações de seu edifício abalada após a 
         conclusão do trabalho. Esta foi a posição em que fui colocado por uma carta do 
         Sr. Bertrand Russell, exatamente quando a impressão deste volume estava 
         chegando ao fim... \\
         ... O que está em questão não é apenas minha maneira particular de 
         estabelecer a aritmética, mas se a aritmética pode receber uma base lógica. 
         Mas vamos ao que interessa. O Sr. Russell encontrou uma contradição que agora 
         deve ser declarada..."}
         \begin{flushright}
            The Frege Reader. M. Beaney, ed., Blackwell, 1977. pp. 279-80.
         \end{flushright}
      \end{Center}
   \subsection{Princípio do Círculo Vicioso}
      A reação imediata de Russell a esse paradoxo foi propor um princípio 
      abrangente projetado para excluir não apenas os problemas específicos 
      levantados pelo seu paradoxo , mas também todos os paradoxos 
      auto-referenciais e mentirosos envolvendo totalidades semelhantes.
      \begin{Center}
         \textit{"Isso nos leva à regra: 'Tudo o que envolve toda a coleção 
         não deve ser parte da coleção;' ou, inversamente: 'Se, desde que uma 
         determinada coleção tivesse um total, ela tivesse membros apenas definíveis 
         em função desse total, então a dita coleção não tem total.'\\
         "... Quando digo que uma coleção não tem total, quero dizer que declarações 
         sobre todos os seus membros são bobagens..."}
         \begin{flushright}
            Bertrand Russell, "Mathematical Logic as Based on the Theory of Types,"\\ 
            American Journal of Mathematics, 30, 1908 pp. 222-262. Em p. 225.
         \end{flushright}
      \end{Center}
      Juntamente com Alfred North Whitehead nomeia o princípio:
      \begin{Center}
         \textit{"O princípio que nos permite evitar totalidades ilegítimas pode 
         ser enunciado da seguinte forma: 'O que envolve toda a coleção não deve 
         ser parte da coleção'; ou, inversamente: 'Se, desde que uma determinada 
         coleção tivesse um total, ela teria membros apenas definíveis em termos 
         desse total, então a referida coleção não tem total.' Chamaremos isso de 
         'princípio do círculo vicioso', porque nos permite evitar os círculos 
         viciosos envolvidos na suposição de totalidades ilegítimas. Argumentos 
         que são condenados pelo princípio do círculo vicioso serão chamados de 
         'falácias do círculo vicioso'. "}
         \begin{flushright}
            Bertrand Russell and Alfred North Whitehead, Principia Mathematica.\\
            Cambridge: at the University Press, 1910. Em p. 40.
         \end{flushright}
      \end{Center}
      O princípio exclui imediatamente a existência das estruturas que cercam o 
      paradoxo de Russell. O conjunto de todos os conjuntos é uma entidade negada e 
      falar de seus membros é impedido como absurdo.

   \subsection{A teoria dos tipos de Russell}
      O princípio do círculo vicioso escapa ao paradoxo de Russell afirmando 
      negativamente o que não podemos fazer com conjuntos. E sobre o relato positivo 
      de conjuntos? Russell embarcou no projeto de reformar os fundamentos da teoria 
      e da lógica dos conjuntos para acomodar esse princípio. A resposta de Russell 
      veio na forma de sua "teoria dos tipos". A ideia principal é que os conjuntos 
      vivem em uma hierarquia. Isso já remove imediatamente o problema: o paradoxo 
      se resolve pois a noção do conjunto ser membro de si mesmo, ou não membro de 
      si mesmo, viola a hierarquia.\newline
      A teoria é colocada dentro dos limites formais da lógica simbólica. Essa é uma 
      lógica que fala de indivíduos; seus predicados, ou seja, propriedades; e 
      propriedades das propriedades; e assim por diante. Os conjuntos surgem como 
      subproduto, como extensão de predicados. Ou seja, são as coleções de coisas 
      que carregam a propriedade.\newline
      As entidades do tipo zero são indivíduos como uma maçã (vermelha) ou uma 
      pêra (verde). As entidades do tipo um são propriedades de indivíduos, 
      como "é vermelho" ou "é verde". O que os torna do tipo um é que eles podem 
      ser predicados apenas de entidades do tipo zero. Isso significa que, por 
      exemplo $$\textrm{vermelho}(\textrm{maçã})$$ significando "a maçã carrega 
      a propriedade vermelho" é permitido. No entanto, 
      $\textrm{vermelho}(\textrm{verde})$ não é permitido, pois vermelho e verde 
      são entidades do tipo um.\newline
      As entidades do tipo dois são propriedades das entidades do tipo um. 
      "é uma cor" é uma entidade de tipo dois, pois pode ser predicado de entidades 
      de tipo um. Nós podemos ter $$\textrm{cor}(\textrm{vermelho})$$ e 
      $$\textrm{cor}(\textrm{verde}).$$ A classificação dos tipos continua 
      indefinidamente e torna-se mais complicada. Conjuntos (ou "classes" no sistema 
      de Russell) surgem como o que se conhece como "extensões" de predicados.
      \newline
      Considere todos os objetos que carregam a propriedade "vermelha". Temos 
      que eles formam o conjunto de coisas vermelhas. Agora, fazendo a extensão 
      da propriedade "é uma cor", podemos, então, formar um conjunto de nível 
      superior: o conjunto de cores.\newline
      Esta é a única maneira pela qual os conjuntos podem ser formados. A estrutura 
      hierárquica resultante de conjuntos nos impede até mesmo de perguntar se um 
      conjunto é membro de si mesmo. Pois um conjunto herda o tipo do predicado que 
      o define. Um conjunto só pode ser formado a partir de conjuntos de um tipo 
      inferior. Por exemplo, o conjunto de cores é do tipo dois. Ele deriva esse tipo 
      do predicado do tipo dois "é uma cor". Sua extensão é o conjunto de cores: 
      $$\{\textrm{vermelho}, \textrm{verde}, \textrm{azul}, ...\}.$$ O predicado é 
      aplicado a predicados do tipo um, vermelho, verde, etc.
      $$\textrm{cor}(\textrm{vermelho}), \textrm{cor}(\textrm{verde}), ...$$
      O predicado "é uma cor" não pode ser aplicado a si mesmo, pois então 
      estaríamos tentando aplicar um predicado de tipo dois a um predicado de tipo 
      dois. Isto é, não podemos ter $\textrm{cor}(\textrm{cor})$. Não podemos incluir 
      o conjunto de cores como membro do conjunto de cores, juntamente com as cores 
      específicas, como vermelho e verde.\newline
      Paradoxo de Russell? Na teoria dos tipos de Russell? Nada disso.
      $$\textrm{Russell}(\textrm{Russell})\quad \Rightarrow\Leftarrow$$ Pronto, 
      está resolvido!
