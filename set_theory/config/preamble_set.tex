% PACOTES BASE
   \usepackage[portuguese]{babel}
   \usepackage[utf8]{inputenc}
   \usepackage[T1]{fontenc}
   \usepackage{amsmath, amsfonts, mathtools, amsbsy, amssymb}
   \usepackage{amsthm}


% PACOTES ADICIONAIS
   \usepackage{newpxtext,eulerpx} % font pessoal
   \usepackage[margin=2cm]{geometry}
   \usepackage[shortlabels]{enumitem}
   \usepackage[hidelinks]{hyperref}
   \usepackage{graphicx}
   \usepackage{tcolorbox}
   \usepackage[framemethod=default]{mdframed}
   \usepackage{tikz}
   \usepackage{csquotes}
   \usepackage{ragged2e}
   \usetikzlibrary{matrix,arrows}
   \usetikzlibrary{calc,positioning,shapes.geometric}
   \usepackage[backend=bibtex,natbib=true,style=numeric,sorting=none]{biblatex}
   \addbibresource{./refs/book_ref.bib}

% COMANDOS ATALHOS
   % conjuntos numéricos
      \newcommand{\N}{\mathbb{N}}
      \newcommand{\Z}{\mathbb{Z}}
      \newcommand{\Q}{\mathbb{Q}}
      \newcommand{\R}{\mathbb{R}}
      \newcommand{\C}{\mathbb{C}}
      \newcommand{\I}{\mathbb{I}}
      \newcommand{\Nnz}{\mathbb{N}^*}
      \newcommand{\Zm}{\mathbb{Z}_+}
      \newcommand{\Rnz}{\mathbb{R}^*}

   % outros atalhos
      \newcommand{\LinhaR}{\rule{\linewidth}{0.5mm}}
      \renewcommand{\thefootnote}{\roman{footnote}} % notas serão em algarismo romano
      \newcommand{\subscript}[2]{$#1 _ #2$}
      \newcommand{\Equivn}[1]{\underset{#1}{\equiv}}
      \newcommand{\fonta}{\fontfamily{lmss}\selectfont }
      \newcommand{\fontr}{\fontfamily{lmtt}\selectfont }
      \newcommand{\highl}{\color{red}}

\global\mdfdefinestyle{exampledefault}{%
linecolor=lightgray,linewidth=1pt,%
leftmargin=1cm,rightmargin=1cm,
}
\newenvironment{mymdframed}[1]{%
\mdfsetup{%
frametitle={\colorbox{white}{\,#1\,}},
frametitleaboveskip=-\ht\strutbox,
frametitlealignment=\raggedright
}%
\begin{mdframed}[style=exampledefault]
}{\end{mdframed}}

\newtcolorbox{mybox}{colback=gray!25!white,colframe=gray!75!black}

\setlength\parindent{0pt}
\setcounter {secnumdepth}{-1}
\renewcommand\qedsymbol{$\blacksquare$}

\theoremstyle{definition}
\newtheorem{definition}{Definição}
\newtheorem{exmp}{Exemplo}
\newtheorem{stat}{Proposição}

\newtheorem{theorem}{Teorema}
\newtheorem{corollary}{Corolário}[theorem]
\newtheorem{lemma}[theorem]{Lema}

\graphicspath{{./img/}}