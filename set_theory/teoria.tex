\section{Teoria Axiomática [Conjuntos]}
\subsection{Teoria ZF}
Condições estabelecidas da teoria:
\begin{enumerate}
  \item Todos os objetos da teoria são conjutos. Dizemos que \textit{todos os elementos de conjuntos serão conjuntos}, e assim sucessivamente, mas paramos ao chegarmos no conjunto vazio.
  \item Os objetos da teoria não comportarão átomos e nem conjuntos de todos os conjuntos (conjuntos universais), isto é, não comportarão \textit{urelementos}.
\end{enumerate}
A estrutura do método axiomático se estabelece da seguinte forma:
\begin{enumerate}
    \item Linguagem
    \item Axiomas
    \item Teorema - consequência dos axiomas
    \item Regras de inferência da \textit{lógica clássica}
\end{enumerate}

\subsubsection*{1. Composição da linguagem}
A linguagem é dividida em 3 segmentos
\begin{center}
  \textbf{Variáveis}\quad \textbf{Símbolos}\quad \textbf{Fórmulas} 
\end{center}
\begin{enumerate}[a)]
    \item Variáveis: É utilizado tanto letras minúsculas quanto maiúsculas para denotar conjuntos.
    \item Símbolos  
      \subitem $\forall x$: para todo conjunto x...
      \subitem $\exists x$: existe um conjunto x...
      \subitem $\neg$: negação de...
      \subitem $\&$: e
      \subitem $\lor$: ou (acomete uma disjunção)
      \subitem $\rightarrow$: se... então (condicional)
      \subitem $\leftrightarrow$: se e somente se (bicondicional)
    \item Fórmulas: Sequências de símbolos que satisfazem os seguintes itens
        \subitem Se $x$ e $y$ são variáveis, então $x \in y$\ e\ $x = y$ são fórmulas. \textit{Fórmulas atômicas};
        \subitem Se $A$ e $B$ são fórmulas, então $\neg(A), (A) \rightarrow (B), (A) \& (B), (A) \lor (B)$\ e\ $(A) \leftrightarrow (B)$ também são.
        \subitem Se $A$ é uma fórmula e $x$ uma variável, então $\forall x(A)$\ e\ $\exists x(A)$ também são fórmulas;
        \subitem Todas as fórmulas tem uma das formas descritas acima.
\end{enumerate}
\begin{exmp}
    A análise da composição de fórmulas de
    $$\exists B(\forall x(\neg(x \in B))$$
    se da por
\begin{enumerate}[I.]
    \item Existência da fórmula atõmica\ $[(x \in B)]$.
    \item Uma outra fórmula está sendo a negação da fórmula atõmica\ $[(\neg(x \in B)]$
    \item Fórmula quantificando a variável $x$ em relação a negação da fórmula atômica\ $[\forall x()]$
\end{enumerate}
Portanto, temos 3 fórmulas que são $$\neg(x \in B)),\quad \forall x(\neg(x \in B)) \quad e\quad \exists B (\forall x(\neg(x \in B)).$$
\textit{"Existe um objeto $B$ tal que para todo objeto $x$, não é verdade que $x$ pertence a $B$}."
\end{exmp}
\begin{mymdframed}{Observação (Fórmulas lógicas)}
    \begin{itemize}
        \item\textbf{\underline{Ocorrência de variável:}} cada aparecimento de uma variável.
            Sendo $$x \in y\ \&\ \exists x (y = x)$$
            temos duas ocorrências da variável $y$ e duas\footnote[1]{Alguns autores colocam o $x$, em $\exists x$, como uma ocorrência.} ocorrências da variável $x$.
        \item \textbf{\underline{Escopo de uma variável:}} uma variável $y$ está no escopo de uma variável $x$ se $y$ aparece em uma fórmula do tipo $\forall x (\ \ )$\ ou \ $\exists x (\ \ )$.
            Sendo $$x \in y\ \& \ \exists x (y = x)$$
            $y$ está no escopo de $x$ e $x$ está no escopo de si mesmo.
        \item \textbf{\underline{Variável livre:}} uma variável é livre se não está no escopo dela mesma.
            Sendo $$x \in y\ \& \ \exists x (y = x)$$
            $x$ e $y$ que estão fora da subfórmula são variáveis livres. $y$ que está dentro da subfórmula é uma variável livre, mas, dentro da subfórmula, $x$ não é uma variável livre por está dentro do próprio escopo.
    \end{itemize}
\end{mymdframed}
\subsubsection*{2. Axiomas da teoria}  
\begin{itemize}
    \item \textbf{Axiomas que garantem a existência de certos conjuntos específicos}
        \subitem Axioma do vazio
        \subitem Axioma do infinito    
    \item \textbf{Axiomas que nos permitem construir novos conjuntos}
        \subitem Axioma do par
        \subitem Axioma da união
        \subitem Axioma das partes
        \subitem Axioma da escolha
        \subitem Axioma da separação
        \subitem Axioma da substituição
    \item \textbf{Axiomas que dizem respeito ao "comportamento" dos conjuntos}
        \subitem Axioma da regularidade 
        \subitem Axioma da extensão
\end{itemize}
Nas partes 3 e 4 da estrutura da Teoria ZF, que são, respectivamente, os teoremas e as regras de inferência da lógica clássica, surgem dos Axiomas que serão apresentados a seguir; não sendo necessário declará-los a priori.

\subsection{Axiomas}
\subsubsection{Axioma da extensão (ou da unicidade)}
\begin{stat}
    Se dois conjuntos $x$ e $y$ possuem exatamente os mesmos elementos, então eles são iguais.
\end{stat}
    $$\forall x\ \forall y (\forall z (z \in x\ \leftrightarrow \ z \in y)\ \rightarrow x = y)$$
    \begin{exmp}
        Se $x = \{1, 2, 3\}$, $y = \{3,1,2\}$ e $z = \{1,1,2,3,3,3\}$, então $x = y = z$.
    \end{exmp}
\subsubsection{Axioma do vazio}
\begin{stat}
    Existe um conjunto sem elemento algum.\footnote[2]{Posteriormente, surgirá a necessidade de verificar a dispensabilidade deste axioma.}
\end{stat}    
    $$\exists x \forall y\ \neg(y \in x)$$
Utiliza-se $\varnothing$ para denotar este conjunto do Axioma do vazio. Para isto, devemos que assegurar que
\begin{enumerate}
    \item Existe um conjunto com tal propriedade
    \item Esse conjunto é único (não pode-se atribuir o mesmo símbolo para duas coisas distintas)
\end{enumerate}
    \begin{definition}[1. Relação de inclusão]
        Dizemos que $x$ está contido em $y$ e escrevemos $x \subseteq y$ se todo elemento de $x$ é um elemento de $y$. 
        $$ x \subseteq y\ \ \leftrightarrow\ \ \forall z (z \in x\ \ \rightarrow\ \ z \in y)$$
        Com essa notação, podemos reescrever o \textbf{axioma da extensão}: $$(A \subseteq B)\ \&\ (B \subseteq A)\ \ \rightarrow\ \ A = B$$
        Também escrevemos $x \subset y$ para dizer que $x \subseteq y\ \ \&\ \ x \neq y$.
    \end{definition}
\begin{theorem}\label{T1}
    O conjunto vazio está contido em qualquer conjunto.
    $$\forall x (\varnothing \subseteq x)$$
\end{theorem}
\begin{proof}
  Supondo que exista x tal que o conjunto vazio não esteja contido em x e que existe y tal que y pertence ao conjunto vazio mas não a x, temos, com efeito: 
  $$\exists x(\varnothing \nsubseteq x)\ \ \&\ \ \exists y(y \in \varnothing\ \&\ y \notin x).$$
Temos uma contradição à própria definição de conjunto vazio.
\end{proof}

\begin{theorem}\label{T2}
    Existe um único conjunto vazio.    
\end{theorem}
\begin{proof}
A existência é garantida pelo Axioma do vazio. Com efeito, supondo que existam conjuntos vazios $x$ e $y$, com $x \neq y$. Pelo axioma da extensão deve então existir um elemento $z$ tal que
\begin{alignat*}{3}
    & z \in x  \quad & mas \quad & z \notin y\quad ou\\
    & z \in y  \quad & mas \quad & z \notin x.
\end{alignat*}
Em qualquer caso temos uma contradição com o fato de que $x$ e $y$ são vazios.
\end{proof}

\begin{mymdframed}{Observação (Relação de inclusão)}
    É comumente apresentado em escolas que a relação de inclusão se aplica apenas entre conjuntos, enquanto que a relação de pertinência se aplica entre elementos e conjuntos. Isso é impreciso diante da teoria, pois:
    \begin{itemize}
        \item Todos os objetos da nossa teoria são conjuntos. Portanto, nas expressões $x \in y$\ \ e\ \ $x \subseteq y$, tanto $x$ quanto $y$ são conjuntos;
        \item Podem existir conjuntos cujos elementos são conjuntos.
    \end{itemize}
    Suponde que exista o conjunto $\{\varnothing\}$, ou seja, um conjunto cujo único elemento é $\varnothing$. Em consideração, temos
    \begin{itemize}
        \item $\varnothing \in \{\varnothing\}$, isto é, $\varnothing$ é elemento do conjunto $\{\varnothing\}$
        \item $\varnothing \subseteq \{\varnothing\}$, de acordo com o Teorema\ref{T1}.
    \end{itemize}
\end{mymdframed}
\subsubsection{Axioma do Par}
\begin{stat}
    Sejam $x$ e $y$ conjuntos quaisquer. Então existe um conjunto $x \in z$\ \ e\ \ $y \in z$.
\end{stat}
    $$\forall x, y\ \exists z (x \in z\ \&\ y \in z)$$
\begin{center}
Aqui, dizemos que $z$ é um par não ordenado, pois, pelo axioma da extensão, \\
os conjuntos $\{x, y\}$\ e\ $\{y, x\}$\ são iguais.
\end{center}
\begin{exmp}
    No caso em que $x=y$: \\
    temos $\{x, x\} = \{x\}$, pelo axioma da extensão.
\end{exmp}

\subsubsection{Axioma da União}
\begin{stat}
    Para todo conjunto $A$, existe um conjunto $B$ cujos elementos são exatamente os elementos dos elementos de $A$.
\end{stat}
    B será denotado por $\bigcup\limits_{X \in A}^{} A$
\begin{itemize}
  \item todo $X$, que é elemento de $A$, é subconjunto de $B$
  \item todo $Y$, que é elemento de $B$, é elemento de algum elemento de $A$
\end{itemize}
Neste axioma, temos, também que
\begin{itemize}
  \item Caso $A$ seja um vazio, $B$ será o conjunto vazio
  \item Caso $A$ seja finito, $B$ será uma união finita
  \item Caso $A$ seja infinito, $B$ será uma união infinita.
\end{itemize}

\begin{definition}
    A \underline{união generalizada} de um conjunto $A$ é o conjunto formado pelos elementos dos elementos de $A$.
$$\forall A\,\exists B\,\forall x\,(x\in B\ \leftrightarrow\ \exists c\,(c\in A\ \&\ x\in c))$$
\end{definition}
\begin{exmp}
    Seja um conjunto $A = \{\{1,\ 2\},\ \{3,\ 4\},\ \{5,\ 6\}\}$.\\
    Então, a união generalizada se realiza como $\bigcup A = \{1,\ 2,\ 3,\ 4,\ 5,\ 6,\}$.\\
    O conjunto $A$ possui 3 elementos que são conjuntos que possuem elementos que estão em $A$.
\end{exmp}

\begin{mymdframed}{Observação}
Não existe um axioma semelhante para interseções, pois não é possível definir uma interseção vazia. A redação de $\bigcap\limits_{X \in \varnothing}^{} X$ compreende o \textit{Universo de von Neumann}, o que nos levaria para caminhos desnecessários (por ora) ao estudo mais avançado da Teoria dos conjuntos.
\end{mymdframed}

\subsubsection{Axioma das partes} 
\begin{stat}
    \textit{Para todo conjunto $x$, existe o conjunto $y$ dos subconjuntos de $x$:}
\end{stat}
    $$\forall x \exists y \forall z [z \in y \leftrightarrow z \subseteq x]$$
O conjunto definido pelo axioma também é único. Sendo assim, podemos definir o seguinte:
\begin{definition}
    Definimos o \textit{conjunto das partes}, ou simplesmente \textit{power set}, como o conjunto de todos os subconjuntos de $x$, e o denotamos por $\mathcal{P}(x)$.
\end{definition}
\begin{exmp}
    Seja um conjunto $A = \{1,2\}.$\\
    Pelo axioma das partes, $$(z \in  \mathcal{P}A\ \leftrightarrow\ z \subseteq A)$$
\begin{alignat*}{3}
    & \varnothing \subseteq A\\
    & \{1\} \subseteq A\\
    & \{2\} \subseteq A\\
    & \{1,2\} \subseteq A.
\end{alignat*}
Logo,\quad $\mathcal{P}A = \{\varnothing,\ \{1\},\ \{2\},\ \{1,\ 2\}\}$
\end{exmp}

\subsubsection{Axioma da separação}
Queremos ser capazes de definir um conjunto através de uma fórmula lógica. Para evitar paradoxos, como o paradoxo de Russel, temos que especificar o conjunto sobre qual nossa fórmula será elaborada.
\begin{stat}
    Portanto, para cada fórmula $P(x)$, para todo conjunto $y$, existe o conjunto daqueles elementos que pertencem a $y$ e satisfazem a fórmula $P$.
\end{stat}
\begin{exmp}
    Seja um conjunto $y = \{1,2,3,4,5,6,7,8,9\}$.

    $P(x): x$ é par.

    $Z = \{x; x \in y\ \&\ P(x)\} = \{2,4,6,8\}$
\end{exmp}
\begin{definition}
    Para cada fórmula $P$ tal que $z$ não ocorre livre, a seguinte fórmula é um axioma:
    $$\forall y \exists z \forall x [x \in z\ \leftrightarrow\ x \in y\ \&\ P(x)]$$    
\end{definition}
\textit{"para todo $y$ existe $z$ tal que para todo $x$, $x$ pertence a $z$ iff $x$ pertence a $y$ e vale a fórmula $P(x)$."}\\
O conjunto $z$ será denotado assim:
$$z = \{x: x \in y\ \&\ P(x)\}.$$
Ou assim:
$$z = \{x \in y: P(x)\}.$$
Por que $z$ não pode ocorrer \underline{livre} em $P$ na fórmula
$$\forall y \exists z \forall x [x \in z\ \leftrightarrow\ x \in y\ \&\ P(x)] ?$$
Se permitirmos que a variável que \underline{define} o conjunto ocorra livre em $P$, poderíamos tomar \\ $P(x) = \highl{x \notin z}$:
$$\forall y \exists z \forall x [x \in z\ \leftrightarrow\ x \in y\ \&\ \highl{x \notin z}\color{black}].$$
Tomando $y = \{\varnothing\}$\ e\ $x = \varnothing$, temos que $x \in y$ é verdadeiro. Assim,
$$(x \in z\ \leftrightarrow\ \highl{x \notin z} \color{black})$$
é uma \underline{contradição}. 

\subsubsection*{Interseção}
Podemos utilizar o axioma da separação para definir a interseção de dois conjuntos.

Sejam $A$ e $B$ dois conjuntos.

Então, pelo axioma da separação, existe $z = \{x: x \in A\ \&\ P(x)\}$

Tomando $\color{brown}{P(x) = x \in B}$, temos
$$\color{violet}{z = \{x: x \in A\ \&\ x \in B \} = A \cap B}$$

\subsubsection*{Diferença e conjuntos vazios}
Da mesma forma, podemos definir a diferença entre dois conjuntos.

Sejam $A$ e $B$ dois conjuntos.

Então, pelo axioma da separação, existe $z = \{x: x \in A\ \&\ P(x)\}$

Tomando $\color{brown}{P(x) = x \notin B}$, temos
$$\color{violet}{z = \{x: x \in A\ \&\ x \notin B \} = A - B}$$
Se tomarmos $\color{brown}{P(x) = x \neq x}$, obtemos o conjunto vazio.

\subsubsection*{Interseção arbitrária de conjuntos}
\begin{theorem}
    Dado um conjunto não vazio $x$, existe o conjunto $y = \bigcap x$, formado por todos os elementos que pertencem simultaneamente a todos os elementos de $x$.
$$\bigcap x = \{v: v\ \textrm{pertence a todos os elementos de}\ x\}.$$
\end{theorem}
\begin{exmp}
    Seja um conjunto $x = \{\{1,2,3\},\ \{2,3,4\},\ \{2,3,4,5\}\}.$\\
    Existe o conjunto interseção: $\bigcap x = \{2,3\}.$

    Formalmente, temos $$y = \{v: v \in z\ \&\ \forall w (w \in x\ \rightarrow\ v \in w)\}.$$
\end{exmp}
\subsubsection{Axioma da regularidade}
    \begin{stat}
        Para todo conjunto $x$ não vazio existe $y \in x$ tal que $x \cap y = \varnothing$.
    \end{stat}
        $$ \forall x [x \neq \varnothing\ \rightarrow\ \exists y (y \in x\ \&\ x \cap y = \varnothing)]$$
Ou seja, esse axioma garante que não exista uma sequência infinita descrescente na relação de pertinência. Não há como ficar preso numa sequência infinita decrescente, pois certa hora irá encontrar o conjunto vazio.
\newpage
\begin{theorem}
    Não existem conjuntos $x$ e $y$ tais que $x \in y$\ e\ $y \in x$.
\end{theorem}
\begin{proof}
    Sejam $x$ e $y$ conjuntos quaisquer.    
    
    Vamos demosntrar que ou $x \notin y$\ ou\ $y \notin x$. Pelo \textbf{axioma do par}, existe $B = \{x,y\}$ não vazio.
    
    Pelo \textbf{axioma da regularidade}, $B \neq \varnothing\ \rightarrow\ \exists z (z \in B\ \&\ z \cap B = \varnothing).$
    
    Mas $z \in B$ implica que $z = x$\ ou \ $z = y$.
    
    Se $z=x$, então $z \cap B\ =\ x \cap B\ =\ \varnothing$, logo, $y \notin x$,

    (pois caso contrário, o resultado da interseção conteria o $y$).

    Se $z=y$, então $z \cap B\ =\ y \cap B\ =\ \varnothing$, logo, $x \notin y$.
\end{proof}
\begin{mymdframed}{Nota}
A formulação, dada por von Neumann (1925), em lógica de primeira ordem, é seguida por
$$\forall A(\exists B(B\in A)\rightarrow \exists B(B\in A\land \neg \exists C(C\in A\land C\in B)))$$
\textit{Todo conjunto diferente do conjunto vazio possui um elemento que é totalmente disjunto dele.}\\
Sem esse axioma, o comportamento possível de infundados conjuntos surgem. São denominados \textbf{hiperconjuntos}. Um comportamento típico seria $y \in y$, então $y$ é um hiperconjunto. Em particular, não é adequado ter a violação desse axioma, apenas para versões da teoria dos conjuntos que possam se beneficiar de tal comportamento.
\end{mymdframed}