\section{Conjuntos}
\subsection{Pertinência e Inclusão}

Quando um objeto (elemento) $x$ pertence a um conjunto $A$: $$x \in A$$
Dois conjuntos iguais, onde possuem exatamente os mesmos elementos (um dado conjunto está totalmente contido no outro) e dois conjuntos que possuem parcialmente os objetos do outro (um dado conjunto está parcialmente contido no outro): 
$$A \subseteq B\ \iff\ A=B\quad \quad \textrm{(1)}$$  
$$A \subset B\ \iff\ A \neq B\quad \quad \textrm{(2)}$$
(1)\ \ \ $\forall x; x \in A$ \& $x \in B$. \\
(2)\ \ \ $x,y; ((x \in A)\ \&\ (x,y \in B))\ \lor\ ((y \in A)\ \&\ (x,y \in B))$.

\subsection{União e Interseção}
A união $x \cup y$ dos conjuntos $x$ e $y$ é definida sendo o conjunto de todos os elementos de $x$ e $y$.

Logo,\quad $(x \cup x = x)$,\ \ \ $(x \cup y = y \cup x)$\ \ \ e\ \ \ $((x \cup y) \cup z = x \cup (y \cup z))$.

De modo geral, dados $n$ conjuntos $A_{1}, A_{2}, \dots, A_{n}$ a sua união é
\[
\bigcup_{i=1}^{n} A_{i} = \{x \in U \mid x \in A_{1}\quad ou\quad x \in A_{2}\quad ou\quad \dots \quad x \in A_{n}.\}
\]
\\
A interseção $x \cap y$ é o conjunto dos objetos que $x$ e $y$ possuem em comum.

Logo,\ \ $x \cap x = x$,\ \ \ $x \cap y = y \cap x$\ \ \ e\ \ \ $(x \cap y) \cap z = x \cap (y \cap z)$.

De modo geral, a interseção dos conjuntos $A_{1}, A_{2}, \dots, A_{n}$ é
\[
\bigcap_{i=1}^{n} A_{i} = \{x \in U \mid x \in A_{1},\ x \in A_{2},\  \dots\ , x \in A_{n}.\}
\]

\textit{Propriedade Associativa}: $(x \cup y) \cup z = x \cup (y \cup z)$.
\begin{proof}
  Tomemos os conjuntos $A$, $B$ e $C$.\\
  Provemos que $(A \cup B) \cup C = A \cup (B \cup C).$\\
  $(A \cup B) \cup C = \{x \mid (P(x) \lor Q(x)) \lor R(x) \} = \{x \mid P(x) \lor (Q(x) \lor R(x)) \} = A \cup (B \cup C).$
\end{proof}

\textit{Propriedade Comutativa}: $x \cup y = y \cup x$.
\begin{proof}
  Tomemos os conjuntos $A$ e $B.$\\
  Provemos que $A \cup B$ = $B \cup A.$\\
  $A \cup B = \{x \mid x \in A\ \ \lor\ \ x \in B \} = B \cup A.$
\end{proof}

\textit{Propriedade Distibutiva}: $x \cup (y \cap z) = (x \cup y) \cap (x \cup z)$\ \ ou\ \ $x \cap (y \cup z) = (x \cap y) \cup (x \cap z)$.
\begin{proof}
  Tomemos os conjuntos $A, B$ e $C.$\\
  Provemos que $A \cup (B \cap C) = (A \cup B) \cap (A \cup C).$
    \begin{alignat*}{3}
    & A \cup (B \cap C) &=& \{x \mid P(x) \lor (Q(x) \land R(x))\}\\
    & &=& \{x \mid (P(x) \lor Q(x)) \land (P(x) \lor R(x))\}\\
    & &=& \{x \mid (P(x) \lor Q(x)) \cap (P(x) \lor R(x))\}\\
    & &=& (A \cup B) \cap (A \cup C).
    \end{alignat*}
\end{proof}

\subsection{Conjunto universal}
\begin{stat}
  Não existe o conjunto universal na Teoria Zermelo-Fraenkel
\end{stat}
\begin{theorem}
  Não existe o conjunto universal.
\end{theorem}
\begin{proof}
  Suponha por absurdo que $V$, um conjunto universal, exista. Isto é, $$\exists V \forall x (x \in V).$$
  Pelo \textbf{axioma da separação} definimos $A$: $$A = \{x:x \in V\ \&\ x \notin x\}$$
  Note que:
    \begin{itemize}
      \item  Se $A \in A$, então $A \in V\ \&\ A \notin A.$ \quad (contradição)
      \item  Se $A \notin A$, então $A \notin V$\ ou $A \in A.$ Mas não podemos ter $A \in A.$
    \end{itemize}
Portanto, $A \notin V$, contrariando o fato de $V$ ser universal.
\end{proof}