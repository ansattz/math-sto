\section{Sequências}
   \subsection{Sequências infinitas}
      Sequências são um tipo de especial de função. Considere uma função de variável inteira, ou seja $$a_{n} := a(n): \Z \to A; n\ \mapsto\ a_{n}.$$
      O domínio de uma sequência é sempre o conjunto dos inteiros. A imagem da sequência depende do contexto, pois o contradomínio pode ser um subconjunto do conjunto $\R$, $\C$ ou um espaço topológico. 
      De qualquer forma, a imagem é geralmente denotada por $a_{n}$.
      \subsubsection{Sequências infinitas unilaterais}
          São sequências onde o domínio pode ser sempre o conjunto $\N$. Considere a aplicação $f: N \to A; n \mapsto f(n).$ Uma sequência infinita unilateral seria algo do tipo $$\left\{a_{n}\right\}^{\infty}_{n \in \N}.$$
         \begin{exmp}
            Considere a sequência $\left\{\cos{\dfrac{n\pi}{6}}\right\}^{\infty}_{n=0}$. Então, a imagem é $\left\{1,\dfrac{\sqrt[]{3}}{2}, \dfrac{1}{2}, ...\right\}.$
         \end{exmp}
      
      \subsubsection{Sequências bi-infinitas}
         Sequências bi-infinitas são sequências do tipo $$\left\{a_{n}\right\}^{\infty}_{n=-\infty}.$$
         \begin{exmp}
            Considere $\left\{4n\right\}^{\infty}_{n=-\infty}$. Então, temos $$(..., -16,-12,-8,-4,0,4,8,12,16,...).$$
         \end{exmp}
   \subsection{Sequências monótonas}
      Sequências monótonas são sempre crescentes ou decrescentes. 
         \subsubsection{Sequências monotonicamente crescentes}
                  A sequência $\left\{a_{n}\right\}^{\infty}_{n=1}$ é monotonicamente crescente se e somente se $a_{n+1} \geq a_{n}$, para todo $n \in \N$. 
               Se cada termo consecutivo é estritamente maior que o anterior, então a sequência é \emph{estritamente monotonicamente crescente}.
            \subsubsection{Sequências monotonicamente decrescentes}
               Analogamente, uma sequência é monotonicamente decrescente se a cada termo consecutivo for menor que o anterior. A sequência será \emph{estritamente monotonicamente decrescente} se cada termo for estritamente menor que o anterior.
      
   \subsection{Sequências limitadas}
      \begin{definition}
         Uma sequência $\left\{a_{n}\right\}^{\infty}_{n \in A}$ é limitada quando o conjunto de seus termos é limitado, ou seja, existem $K$ e $M$ contidos em $A$ tais que $K \leq a_{n} \leq M, \forall n \in A$.
         \begin{enumerate}[i.]
            \item Se $\left\{a_{n}\right\}$ é limitada superiormente temos $$a_{n} \leq M,\forall n \in A.$$
            \item Se $\left\{a_{n}\right\}$ é limitada inferiormente temos $$K \leq a_{n}, \forall n \in A.$$
            \item Se $\left\{a_{n}\right\}$ é inferiormente e superiormente
               limitada então $\left\{a_{n}\right\}$ é uma sequência limitada.
               Isto é, ${a_{n}}$ é limitada se existir um $L > 0$, $L \subseteq A$, tal que
               $|a_{n}| \leq L, \forall n \in A$.
         \end{enumerate}
      \end{definition}
   \subsection{Limite de uma sequência}
      Sequências são tipos especiais de funções e com isso podemos investigar seus
      limites. No entanto, temos uma restrição: ${a_{n}}$ está definida para
      valores inteiros de $n$. O único limite que usado será de $a_{n} \to
      +\infty.$
      \begin{definition}[Intuitiva]
         Dada uma sequência ${a_{n}}$ dizemos que o limite de uma sequência é
         $L$ se, equanto $n$ se torna grande, $a_{n}$ comeca a estar
         arbitrariamente perto de $L$. Se enquanto $n \to +\infty$ $a_{n}$ nao
         se aproxima de $L$, então dizemos que o limite nao existe.
      \end{definition}
      \begin{definition}
         Dada uma sequência ${a_{n}}$ dizemos que $\lim_{n\to\infty} a_{n} = L$
         se para todo $\epsilon > 0$ existir um inteiro $k$, tal que $| a_{n} -
         L | < \epsilon$ para todo $n \geq k$.
      \end{definition}
      \begin{theorem}
         Seja ${a_{n}}_{n=n_{0}}$ uma sequência e suponha que $f(x)$ é uma
         função real para a qual $f(n) = a_{n}$ para todos os inteiros $n \geq k$,
         onde $k \geq n_{0}$. Se $$\lim_{x\to\infty} f(x) = L,\quad
         \lim_{n\to\infty} a_{n} = L.$$
      \end{theorem}
      \begin{exmp}
         Seja $a_{n} = \dfrac{5n+1}{6n+7}$. Determine se a sequência
         ${a_{n}}_{n=1}$ tem um limite.
         \begin{proof}
            Como $$\lim_{x\to\infty} \dfrac{5x+1}{6x+7} = \dfrac{5}{6},\quad \textrm{então}\ \lim_{x\to\infty} \dfrac{5n+1}{6n+7} = \dfrac{5}{6}.$$
         \end{proof}
      \end{exmp}
      \begin{mymdframed}{Observação}
         Se $\lim_{x\to\infty} f(x)$ nao existe, $\lim_{n\to\infty} a_{n}$ pode nao existir.
      \end{mymdframed}
      \begin{exmp}
         Considere $a_{n} = \sin(n\pi)$ e $f(x) = \sin(x\pi)$.
         \begin{proof}
            Analisando a sequencia $a_{n}$ vemos que ela resulta em uma lista ordenada de zeros:
            $$\cancelto{0}{\sin(0\pi)}, \cancelto{0}{\sin(1\pi)}, \cancelto{0}{\sin(2\pi)}, \cancelto{0}{\sin(3\pi)}, ...$$
            Como os termos da sequencia sao zeros temos que $\lim_{n\to\infty} a_{n} = 0.$ Mas avaliando para valores reais, i.e, quando consideramos $f(x)$ temos que $\lim_{x\to\infty} f(x)$ nao existe.
            Os valores de $\sin(x\pi)$ para $x\to\infty$ oscilam entre $-1$ e $1$.
         \end{proof}
      \end{exmp}
      \begin{definition}
         Dada duas sequencias ${a_{n}}$ e ${b_{n}}$, a notacao $a_{n} \ll b_{n}$ significa que
         $$\lim_{n\to\infty}\dfrac{a_{n}}{b_{n}} = 0\quad\textrm{e}\quad \lim_{n\to\infty}\dfrac{b_{n}}{a_{n}} = \infty.$$
      \end{definition}
      \begin{theorem}
         Sejam $p$,$q$ reais positivos, com $b>1$. Temos a seguinte relacao:
         $$\ln^{p}(n) \ll n^{q} \ll b^{n} \ll n! \ll n^{n}.$$
      \end{theorem}
      Qualquer potencia de $\ln(n)$ cresce mais \emph{lentamente} do que qualquer potencia de $n$.
      \begin{exmp}
         Seja $a_{n} = \dfrac{\ln^{9}(n)}{n^{1/2}}$. Ache o limite da sequencia $a_{n}$.
         \begin{proof}
            O teorema anterior indica que $\ln^{p}(n) \ll n^{q}$, i.e, $\lim_{n\to\infty}\dfrac{\ln^{p}(n)}{n^{q}} = 0$, para qualquer real positivo $p$ e $q$. Portanto, $\lim_{n\to\infty}\dfrac{\ln^{9}(n)}{n^{1/2}} = 0.$
         \end{proof}
      \end{exmp}
      \begin{exmp}
         Seja $a_{n} = \dfrac{n^{100}+n^{n}}{n!+5^{n}}.$Ache o limite da sequencia $a_{n}$.
         \begin{proof}
            Vamos comecar analisando o numerador. Pelo teorema, temos que $n^{n} \gg n^{q}$, que nesse caso eh $n^{n} \gg n^{100}$. No denominador, novamente pelo teorema, temos que $n! \gg b^{n}$, que nesse caso eh $n! \gg 5^{n}$. Entao, iremos saber da existencia de $\lim_{n\to\infty}a_{n}$ considerando $\lim_{n\to\infty} \dfrac{n^{n}}{n!}.$
            Novamente pelo teorema temos que $n^{n} \gg n!$, assim $\lim_{n\to\infty} a_{n} = \infty.$
            
            Mais precisamente, 
            $$\lim_{n\to\infty}\dfrac{n^{100}+n^{n}}{n!+5^{n}} = \lim_{n\to\infty}\dfrac{n^{n}\left(\cancelto{0}{\lim_{n\to\infty}\left(\dfrac{n^{100}}{n^{n}}\right)}+1\right)}{n!\left(1+\cancelto{0}{\lim_{n\to\infty}\left(\dfrac{5^{n}}{n!}\right)}\right)} = \lim_{n\to\infty} \dfrac{n^{n}}{n!}.$$
         \end{proof}
      \end{exmp}
      \begin{theorem}
         Suponha que $a_{n}$,$b_{n}$ e $c_{n}$ sao sequencias com $$a_{n} \leq b_{n} \leq c_{n}.$$ Se $$\lim_{n\to\infty}a_{n} = L = \lim_{n\to\infty}c_{n}$$, entao $\lim_{n\to\infty}b_{n} = L.$
      \end{theorem}
   \subsection{Propriedades de limites de sequências}
   \begin{stat}[Propriedades aritmeticas]
      Considere duas sequencias ${a_{n}}$ e ${b_{n}}$ tais que $\lim_{n\to\infty} a_{n} = L$ e $\lim_{n\to\infty} b_{n} = M$, então
      \begin{enumerate}[i.]
         \item $\lim_{n\to\infty} (a_{n} + b_{n}) = L + M$;
         \item $\lim_{n\to\infty} (a_{n} - b_{n}) = L - M$;
         \item $\lim_{n\to\infty} (a_{n} \cdot b_{n}) = L \cdot M$;
         \item $\lim_{n\to\infty}\dfrac{a_{n}}{b_{n}} = \dfrac{L}{M}$, $M \neq 0.$
      \end{enumerate}
   \end{stat}
   \subsection{Subsequências}
   \subsection{Sequência de Cauchy}

   \subsection{Convergência de sequências}
