\section{Sequências}
   \subsection{Sequências infinitas}
      Sequências são um tipo de especial de função. Considere uma função de variável inteira, ou seja $$a_{n} := a(n): \Z \to A; n\ \mapsto\ a_{n}.$$
      O domínio de uma sequência é sempre o conjunto dos inteiros. A imagem da sequência depende do contexto, pois o contradomínio pode ser um subconjunto do conjunto $\R$, $\C$ ou um espaço topológico. 
      De qualquer forma, a imagem é geralmente denotada por $a_{n}$.
      \subsubsection{Sequências infinitas unilaterais}
          São sequências onde o domínio pode ser sempre o conjunto $\N$. Considere a aplicação $f: N \to A; n \mapsto f(n).$ Uma sequência infinita unilateral seria algo do tipo $$\left\{a_{n}\right\}^{\infty}_{n \in \N}.$$
         \begin{exmp}
            Considere a sequência $\left\{\cos{\dfrac{n\pi}{6}}\right\}^{\infty}_{n=0}$. Então, a imagem é $\left\{1,\dfrac{\sqrt[]{3}}{2}, \dfrac{1}{2}, ...\right\}.$
         \end{exmp}
      
      \subsubsection{Sequências bi-infinitas}
         Sequências bi-infinitas são sequências do tipo $$\left\{a_{n}\right\}^{\infty}_{n=-\infty}.$$
         \begin{exmp}
            Considere $\left\{4n\right\}^{\infty}_{n=-\infty}$. Então, temos $$(..., -16,-12,-8,-4,0,4,8,12,16,...).$$
         \end{exmp}
   \subsection{Sequências monótonas}
      Sequências monótonas são sempre crescentes ou decrescentes. 
         \subsubsection{Sequências monotonicamente crescentes}
               A sequência $\left\{a_{n}\right\}^{\infty}_{n=1}$ é monotonicamente crescente se e somente se $a_{n+1} \geq a_{n}$, para todo $n \in N$. 
               Se cada termo consecutivo é estritamente maior que o anterior, então a sequência é \emph{estritamente monotonicamente crescente}.
            \subsubsection{Sequências monotonicamente decrescentes}
               Analogamente, uma sequência é monotonicamente decrescente se a cada termo consecutivo for menor que o anterior. A sequência será \emph{estritamente monotonicamente decrescente} se cada termo for estritamente menor que o anterior.
      
   \subsection{Sequências limitadas}
   \subsection{Limite de uma sequência}
   \subsection{Propriedades de limites de sequências}
   \subsection{Subsequências}
   \subsection{Sequência de Cauchy}

   \subsection{Convergência de sequências}
