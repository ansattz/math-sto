% PACOTES BASE
	\usepackage[portuguese]{babel}
	\usepackage[utf8]{inputenc}
	\usepackage[T1]{fontenc}
	\usepackage{amsmath, amsfonts, mathtools, amsbsy, amssymb}
	\usepackage{amsthm}
	\usepackage[makeroom]{cancel}

% PACOTES ADICIONAIS
	\usepackage{newpxtext,eulerpx} % font pessoal
	\usepackage[margin=2cm]{geometry}
	\usepackage[shortlabels]{enumitem}
	\usepackage[hidelinks]{hyperref}
	\usepackage{tcolorbox}
	\usepackage[framemethod=default]{mdframed}
	\usepackage{tikz}
	\usepackage{csquotes}
	\usepackage[backend=bibtex,natbib=true,style=numeric,sorting=none]{biblatex}

	\usetikzlibrary{matrix,arrows}
	\usetikzlibrary{calc,positioning,shapes.geometric}
   \addbibresource{./refs/book_ref.bib}
	
% COMANDOS ATALHOS
	% conjuntos numéricos
		\newcommand{\Z}{\mathbb{Z}}
		\newcommand{\Zm}{\mathbb{Z}_+}
		\newcommand{\N}{\mathbb{N}}
		\newcommand{\Nnz}{\mathbb{N}^*}
		\newcommand{\Q}{\mathbb{Q}}
		\newcommand{\R}{\mathbb{R}}
		\newcommand{\Rnz}{\mathbb{R}^*}
		\newcommand{\C}{\mathbb{C}}

	% outros atalhos
		\newcommand{\LinhaR}{\rule{\linewidth}{0.5mm}}
		\renewcommand{\thefootnote}{\roman{footnote}} % notas serão em algarismo romano
		\newcommand{\subscript}[2]{$#1 _ #2$}
		\newcommand{\Equivn}[1]{\underset{#1}{\equiv}}
		\newcommand{\fonta}{\fontfamily{lmss}\selectfont }
		\newcommand{\fontr}{\fontfamily{lmtt}\selectfont }
		\newcommand{\highl}{\color{red}}

\global\mdfdefinestyle{exampledefault}{%
linecolor=lightgray,linewidth=1pt,%
leftmargin=1cm,rightmargin=1cm,
}
\newenvironment{mymdframed}[1]{%
\mdfsetup{%
frametitle={\colorbox{white}{\,#1\,}},
frametitleaboveskip=-\ht\strutbox,
frametitlealignment=\raggedright
}%
\begin{mdframed}[style=exampledefault]
}{\end{mdframed}}

\newtcolorbox{mybox}{colback=gray!25!white,
colframe=gray!75!black}

\setlength\parindent{0pt}
\setcounter {secnumdepth}{-1}
\renewcommand\qedsymbol{$\blacksquare$}

\newtheoremstyle{break}
	{\topsep}{\topsep}%
	{}{}%
	{\bfseries}{}%
	{\newline}{}%
\theoremstyle{break}
\newtheorem{theorem}{Teorema}
\newtheorem{definition}{Definição}
\newtheorem{exmp}{Exemplo}
\newtheorem{stat}{Proposição}
\newtheorem{corollary}{Corolário}[theorem]
\newtheorem{lemma}[theorem]{Lema}